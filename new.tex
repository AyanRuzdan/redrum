\documentclass{article}
\usepackage{multicol}
\usepackage{geometry}
\geometry{a4paper,
left = 20mm,
right=20mm,
top=20mm,
bottom=20mm,
heightrounded}
\title{Sem - IV}
\author{\textit{study today}}
\date{\textit{chill tomorrow}}
\begin{document}
\maketitle
\hrule
\begin{multicols*}{2}
    \section*{CSE211: COMPUTER ORGANIZATION AND DESIGN}
    \subsection*{Unit 1}
    \subsubsection*{Basics of Digital Electronics}
    Registers, Shift Registers, Introduction to Combinational Circuits, Introduction to Sequential Circuits
    \subsubsection*{Register Transfer and Micro Operations}
    Bus and Memory Transfer, Logic Micro Operations, Shift Micro Operations, Register Transfer Language, Register Transfer, Arithmetic Logic Shift Unit
    \subsection*{Unit 2}
    \subsubsection*{Computer Organization}
    Instruction Codes, Computer Registers, Common Bus System, Computer Instructions, Timing and Control, Instruction Cycle, Memory Reference Instructions, Input-Output and Interrupt
    \subsection*{Unit 3}
    \subsubsection*{Central Processing Unit}
    General Register Organization, Stack Organization, Addressing Modes, Reduced Instruction Set Computer, Complex Instruction Set Computer, Instruction Formats
    \subsection*{Unit 4}
    \subsubsection*{Input-Output Organization}
    Peripheral Devices, Input Output Interface, Data Transfer Schemes, Program Control and Interrupts, Direct Memory Access Transfer and Input-Output Processor, Priority Interrupt, Direct Memory Access Transfer, Input-Output Processor, Modes of Data Transfer, Processor Status Word
    \subsection*{Unit 5}
    \subsubsection*{Memory Unit}
    Memory Hierarchy and Processor v/s Memory Speed, Associative Memory, Memory Management, Memory Hierarchy, Cache Memory, Virtual Memory, Main Memory, Auxiliary Memory
    \subsection*{Unit 6}
    \subsubsection*{Computer Arithmetic}
    Addition and Subtraction Algorithm, Multiplication Algorithm
    \subsubsection*{Introduction to Parallel Processing}
    Pipelining, Characteristics of Multiprocessors, Interconnection Structures, Parallel Processing
    \subsubsection*{Latest Technology and Trends in Computer Architecture}
    Next generation processor architecture, Microarchitecture, Latest Processor for Smartphone or Tablet and Desktop \\
    \hrule
    \section*{CSE310: PROGRAMMING IN JAVA }
    \subsection*{Unit 1}
    \subsubsection*{Introduction to Java}
    History and Features of Java, Java program structure, Writing simple Java class and main() method, Command-Line arguments, Understanding JDK, JRE and JVM
    \subsubsection*{Data In The Cart}
    Using primitive data types, Type Conversion, Keywords, Identifiers, Variables, Access Modifiers, Static Keyword, Wrapper Class
    \subsubsection*{Operators}
    Working with bit-wise arithmetic, Logical and Relational operators, Unary Assignment and Ternary Operator, Operator Precedence
    \subsubsection*{Conditional Statements}
    Using if-else constructs and switch case statements
    \subsection*{Unit 2}
    \subsubsection*{Loops}
    Working with for loop, while loop, do-while loop and for-each loop
    \subsubsection*{Arrays and Enums}
    Fundamental about Arrays, Multi-dimensional Arrays, Array Access and Iterations, Using varargs, Enumerations
    \subsubsection*{OOP Concepts}
    Basics of class and objects, Writing Constructors and Methods, Overloading Methods and Constructors, this keyword, Initializer Blocks
    \subsubsection*{String Class}
    Constructors and Methods of String and String Builder Class
    \subsection*{Unit 3}
    \subsubsection*{Inheritance and Polymorphism}
    Inheritance, Method overriding, super keyword, Object class and overriding toString() and equals() method, Using super and final keywords, instanceof operator
    \subsubsection*{Abstract Class and Interface}
    Abstract Method and Abstract Class, Interfaces, Static and Default Methods, Using Swing Components to demonstrate Inheritance
    \subsection*{Unit 4}
    \subsubsection*{Functional Interface and Lambda Expressions}
    Using Lambda expressions, Implementing Threads using Lambda expressions, Implementing Listener using Lambda expressions
    \subsubsection*{Nested Class}
    Understanding the importance of static and non-static nested classes, Local and Anonymous Class
    \subsubsection*{Utility Classes}
    Working with Dates
    \subsection*{Unit 5}
    \subsubsection*{Exceptions and Assertions}
    Exception Overview, Exception Class Hierarchy and Exception Types, Propagation of Exceptions, Using try, catch and finally for exception handling, Usage of throw and throws, handling multiple exceptions using multi-catch, Autoclose resources with try-with resources statement, Creating custom exceptions, Testing invariants by using assertions
    \subsubsection*{I/O Fundamentals}
    Describing the basics of input and output in Java, Read and Write Data from the console, Using streams to read and write files, Writing and Reading objects using Serialization
    \subsection*{Unit 6}
    \subsubsection*{Collections and Generics}
    Creating a Custom Generic Class, Using the type inference diamond to create and object, Using bounded types and Wild Cards, Creating a Collection by using Generics, Implementing and ArrayList, Implementing TreeSet using Comparable and Comparator interfaces, Implementing a HashMap, Implementing a Deque
    \subsection*{List of Practicals/Experiments}
    \subsubsection*{Exception Handling}
    Program to demonstrate the use of all the keywords used for exception handling and need of assertion
    \subsubsection*{Multithreading}
    Program to implement multithreading using Lambda expressions.
    \subsubsection*{Creating a Java Main Class}
    Program to implement a Java class
    \subsubsection*{Managing Multiple Items}
    Program to demonstrate the use of list of items
    \subsubsection*{Describing Objects and Classes}
    Program to demonstrate the instantiation of class and accessing the attributes using object of class
    \subsubsection*{Manipulating and Formatting the Data in your program}
    Program to demonstrate the uses of String and StringBuilder
    \subsubsection*{Using Inheritance}
    Program to demonstrate the inheritance and its importance using Swing Components
    \subsubsection*{Overriding Methods, Polymorphism, and Static Classes}
    Program to implement polymorphism and using proper access control
    \subsubsection*{Abstract and Nested Classes}
    Program to demonstrate the use of abstract and nested class
    \subsubsection*{Java I/O}
    Program to implement read and write operation using console and file \\
    \hrule
    \section*{CSE316: OPERATING SYSTEMS}
    \subsection*{Unit 1}
    \subsubsection*{Introduction to Operating System}
    Operating System meaning, Supervisor and User Mode, Review of computer organization, Introduction to popular operating systems like UNIX, Windows etc., OS Structure, System Calls, Functions of OS, Evolution of OSs
    \subsubsection*{Process Management}
    PCB, Operations on Processes, Co-operating and Independent Processes, Inter-Process Communication, Process States, Operations on Processes, Process Management in UNIX, Process Concept, Life Cycle, Process and Threads
    \subsection*{Unit 2}
    \subsubsection*{CPU Scheduling}
    Types of Scheduling, Scheduling Algorithms, Scheduling Criteria, CPU Scheduler-preemptive and non-preemptive, Dispatcher, First Come First Serve, Shortest Job First, Round Robin Priority, Multi-level feedback queue, Multiprocessor Scheduling, Real-time Scheduling, Thread Scheduling
    \subsection*{Unit 3}
    \subsubsection*{Process Synchronization}
    Critical Section Problem, Semaphores, Concurrent Processes, Co-operating Processes, Precedence Graph, Hierarchy of Processes, Monitors, Dining Philosopher Problem, Reader-Writer Problem, Producer Consumer Problem, Classical Two Processes and n-process Solutions, Hardware Primitives for Synchronization
    \subsubsection*{Threads}
    Overview, Multithreading Models, Scheduler Activations, Examples of Threaded Programs
    \subsection*{Unit 4}
    \subsubsection*{Deadlocks}
    Deadlock Characterization, Handling of deadlocks, Deadlock Prevention, Deadlock Avoidance and Detection, Deadlock Recovery, Starvation
    \subsubsection*{Protection and Security}
    Need for Security, Security Vulnerability like Buffer Overflow, Trapdoors, Backdoors, Cache Poisoning etc., Authentication-Password based authentication, Password Maintenance and Secure Communication, Application Security, Virus, Program Threats, Goals of Protection, Principles of Protection, Domain of Protection, Access Matrix, Implementation of Access Matrix, System and Network Threats, Examples of attacks
    \subsection*{Unit 5}
    \subsubsection*{Memory Management}
    Logical and Physical Address Space, Swapping, Contiguous Memory Allocation, Paging, Segmentation, Page Replacement Algorithms, Segmentation-simple, multilevel with paging, Page Interrupt Fault, Fragmentation-internal and external, Schemes-Paging-simple and multilevel, Overlays-swapping, Virtual Memory Concept, Demand Paging
    \subsection*{Unit 6}
    \subsubsection*{File Management}
    File Concepts, Access Methods, Directory Structure, File System Mounting and Sharing, Protection, Allocation Methods, Free-space Management, Directory Implementation
    \subsubsection*{Device Management}
    Dedicated, Shared and Virtual Devices, Serial Access and Direct Access Devices, Disk Scheduling Methods, Direct Access Storage Devices, Channels and Control Units
    \subsubsection*{Inter Process Communication}
    Introduction to IPC(Inter Process Communication) Methods, Pipes – popen and pclose functions, Co-processes, Shared Memory, Stream Pipes, FIFOs, Message Queues, Passing File Descriptors, Semaphores \\
    \hrule
    \section*{CSE325: OPERATING SYSTEMS LABORATORY}
    \subsection*{List of Practicals / Experiments}
    \subsubsection*{Process Creation and Handling}
    \begin{itemize}
        \item Creating Processes
        \item Creating Threads
        \item Process Duplication using fork()
        \item Creating Threads using pthread
        \item Environment Variables
        \item Replacing process image using execlp
    \end{itemize}
    \subsubsection*{Inter-Process Communication}
    \begin{itemize}
        \item Pipes, popen and pclose functions
        \item Stream pipes, passing file descriptors
        \item Shared Memory
        \item Message Passing
        \item Remote Procedure Calls
    \end{itemize}
    \subsubsection*{Introduction to Linux}
    \begin{itemize}
        \item Basic Linux Commands: ls, cat, man, cd, touch, cp, mv, rmdir, mkdir, rm, chmod, pwd
        \item System Calls: Read, Write, Open
        \item Lseek
    \end{itemize}
    \subsubsection*{Synchronization}
    \begin{itemize}
        \item Synchronization with Mutexes
        \item Synchronization with Semaphores
        \item Race Condition
    \end{itemize}
    \subsubsection*{Shell Programming}
    \begin{itemize}
        \item Variables
        \item Standard Input/Output Redirection
        \item Shell Arithmetic
        \item Flow Control and Decision Making
    \end{itemize}
    \subsubsection*{File and Directory Management using System Calls}
    \begin{itemize}
        \item File related system calls open, read, write, lseek, close
        \item Directory related system calls opendir, readdir, closedir
    \end{itemize}
    \hrule
    \section*{CSE408: DESIGN AND ANALYSIS OF ALGORITHMS}
    \subsection*{Unit 1}
    \subsubsection*{Foundation of Algorithm}
    Algorithms, Fundamentals of Algorithmic Problem Solving, Basic Algorithm Design Techniques, Analyzing Algorithm, Fundamental Data Structure, Linear Data
    Structure, Graphs and Trees, Fundamentals of the Analysis of Algorithm Efficiency, Measuring of Input Size, Units for Measuring Running Time, Order of Growth, Worst-Case, Best-Case, and Average-Case Efficiencies, Asymptotic Notations and Basic Efficiency Classes, O(Big-oh)-notation, Big-omega notation, Big-theta notation, Useful Property Involving the Asymptotic Notations, Using Limits for
    Comparing Orders of Growth
    \subsection*{Unit 2}
    \subsubsection*{String Matching Algorithms and Computational Geometry}Sequential Search and Brute-Force String Matching, Closest-Pair and Convex-Hull Problem, Exhaustive Search, Voronoi Diagrams, Naiva String-Matching Algorithm, Rabin-Karp Algorithm, Knuth-Morris-Pratt Algorithm
    \subsection*{Unit 2}
    \subsubsection*{Divide and Conquer and Order Statistics}
    Merge Sort and Quick Sort, Binary Search, Multiplication of Large Integers, Strassen's Matrix Multiplication, Substitution Method for Solving Recurrences, Recursion-Tree Method for Solving Recurrences, Master Method for Solving Recurrence, Closest-Pair and Convex-Hull Problems by Divide and Conquer, Decrease and Conquer: Insertion Sort, Depth-First Search and Breadth-First Search, Connected Components, Topological Sort, Transform and Conquer: Presorting, Balanced Search Trees, Minimum and Maximum, Counting Sort, Radix Sort, Bucket Sort, Heaps and Heapsort, Hashing, Selection Sort and Bubble Sort
    \subsection*{Unit 4}
    \subsubsection*{Dynamic Programming and Greedy Techniques}
    Dynamic Programming: Computing a Binomial
    Coefficient, Warshall's and Floyd's Algorithm, Optimal Binary Search Trees, Knapsack Problem and
    Memory Functions, Matrix-Chain Multiplication, Longest Common Subsequence, Greedy Technique
    and Graph Algorithm: Minimum Spanning Trees, Prim's Algorithm, Kruskal's Algorithm, Dijkstra's
    Algorithm, Huffman Code, Single-Source Shortest Paths, All-Pairs Shortest Paths, Iterative
    Improvement: The Maximum-Flow Problem, Limitations of Algorithm Power: Lower-Bound Theory
    \subsection*{Unit 5}
    \subsubsection*{Backtracking and Approximation Algorithms}
    Backtracking: n-Queens Problem, Hamiltonian Circuit Problem, Subset-Sum Problem, Branch-and-Bound: Assignment Problem, Knapsack Problem, Traveling Salesman Problem, Vertex-Cover Problem and Set-Covering Problem, Bin Packing Problems
    \subsection*{Unit 6}
    \subsubsection*{Number-Theoretic Algorithms and Complexity Classes}
    Number Theory Problems: Modular Arithmetic, Chinese Remainder Theorem, Greatest Common Divisor, Optimization Problems, Basic Concepts of Complexity Classes- P, NP, NP-hard, NP-complete Problems\\
    \hrule
    \section*{INT404: ARTIFICAL INTELLIGENCE}
    \subsection*{Unit 1}
    \subsubsection*{Introduction}
    What is Intelligence? What is Artificial Intelligence? Foundations of Artificial Intelligence (AI), History of AI, Basics of AI, AI Problems, AI Techniques, Applications of AI, Branches of AI
    \subsubsection*{Problem Spaces and Search}
    Defining the problem as a state space search, Production Systems, Problem Characteristics, Production System Characteristics, Issues in Designing Search Problems, Breadth First Search, Depth First Search, Bidirectional Search, Iterative Deepening
    \subsection*{Unit 2}
    \subsubsection*{Informed Search Strategies}
    Hueristic Functions, Generate and Test, Hill Climbing, Simulated Annealing, Best First Search, A* algorithm, Constraint Satisfaction
    \subsection*{Unit 3}
    \subsubsection*{Knowledge Representation}
    Representations and Mappings, Approaches in Knowledge Representation, Issues in Knowledge Representation, Propositional Logic, Predicate Logic, Procedural versus Declarative Knowledge, Logic Programming, Forward versus Backward Reasoning
    \subsection*{Unit 4}
    \subsubsection*{Statistical Reasoning}
    Probability and Bayes' Theorem, Baeyesian Networks, Dempster-Shafer-Theory, Certainty Factors and Rule-Based Systems
    \subsubsection*{Weak Slot and Filler Structures}
    Semantic Nets, Frames
    \subsubsection*{Weak Slot and Filler Structures}
    Conceptual Dependency, Scripts
    \subsection*{Unit 5}
    \subsubsection*{Game Playing}
    Evaluation Function, Min-Max Problem, Min-Max Search Procedure, Alpha-Beta Cutoffs, Alpha-Beta Pruning
    \subsubsection*{Natural Language Processing}
    Introduction to NLP, NLP Phases, Consutruction of Parse Tree, Spell Checking, Bag of Words Model, Soundex Algorithm, Applications of NLP, Alex, Siri and Cortana
    \subsection*{Unit 6}
    \subsubsection*{Advanced Topics in Artificial Intelligence}
    Definition of Machine Learning, Types of Machine Learning, Supervised Learning, Unsupervised Learning, Reinforcement Learning, Overview of Neural Networks, Overview of Genetic Algorithms, Overview of Fuzzy Logic
    \subsubsection*{Current Trends in AI}
    Augmented Workforce, AI in Cybersecurity, Explainable AI, AI and the metavers, autonomous vehicles \\
    \hrule
    \section*{MTH302: PROBABILITY AND STATISTICS}
    \subsection*{Unit 1}
    \subsubsection*{Basics of Probability}
    Sample Space, Events, Counting Sample Points, Probability of an Event, Additive Rules, Conditional Probability, Multiplicative Rules, Bayes' Rule
    \subsection*{Unit 2}
    \subsubsection*{Random Variables and its Characterization}
    Discrete and Continuous Random Variables and their Distrubutions Functions, Joint Probability Distrubutions, Mean of a Random Variable, Variance and Covariance of Random Variables, Chebyshov's Theorem
    \subsection*{Unit 3}
    \subsubsection*{Special Distributions}
    Bernoulli Process, Binomial Distribution, Negative Binomial and Geometric Distributions, Poisson Distribution and Poisson Process, Gamma and Exponential Distributions, Normal Distribution
    \subsection*{Unit 4}
    \subsubsection*{Central Limit Theorem and Point Estimation}
    Central Limit Theorem, Unbiased Estimators, Consistent Estimator, Maximum Likelihood Estimation
    \subsection*{Unit 5}
    \subsubsection*{Hypothesis Testing}
    Types of Error, Student t-test for Single Mean and Difference of Means, Z-Test for Single Mean and Difference of Means, F-Test, Goodness of Fit, Chi-Square Test
    \subsection*{Unit 6}
    \subsubsection*{Correlation and Regressions}
    Scatter Plots, Coefficient of Correlation, Coefficient of Correlation for Bivariate data and Probability Distribution, Spearman's Rank Correlation Coefficient, Linear Regression, Properties of Regression Coefficients, Fitting of a Curve  \\
    \hrule
    \section*{PEA307: ADVANCED ANALYTICAL SKILLS-I}
    \subsection*{Unit 1}
    \subsubsection*{Number System}
    Factors, Factorials, Unit Digit Calculation, Remainder Properties, Advanced Number Based Questions, HCF and LCM
    \subsubsection*{Average}
    Inclusion and Exclusion related questions, Missing Number Questions, Questions based on cricket
    \subsubsection*{Linear Quadratic Equations}
    Statement based questions, Roots of the Quadratic Equation
    \subsection*{Unit 2}
    \subsubsection*{Percentage}
    Percentage to Fraction, Population Change in Percentage, Percente Increase and Decrease, Questions based on Examination and Election
    \subsubsection*{Profit Loss Discount}
    Concepts of Cost Price Selling Price and Marked Price, Successive Discount, Questions based on Selling an Article and Interchanging its values, Dishonest Seller related questions
    \subsubsection*{Simple and Compound Interest}
    Concepts of Interest Calculations, Questions based on the difference between CI and SI, EMI Calculations
    \subsection*{Unit 3}
    \subsubsection*{Logical Reasoning}
    Flow Chart
    \subsubsection*{Analytical Reasoning}
    Coding and Decoding, Number, Alphabet Series Questions, Language Coding Questions
    \subsection*{Unit 4}
    \subsubsection*{Ratio and Proportions}
    Questions based on addition, differnce and product, questions based on Income and Expenditure, questions based on Coins and Rupees, Problems based on Ratio and Proportion and Ages, problems based on Partnerships and Profit Sharing
    \subsubsection*{Alligations and Mixtures}
    Conceptual Knowledge of Alligation and Mixtures, Problems based on Alligation and Mixtures
    \subsection*{Unit 5}
    \subsubsection*{Permutation}
    Counting Method, Numerical Permutation (formation of numbers and sum of numbers), Alpha Permutation (Rearrangement of words and Rank of Words), Logical Permutation, Linear and Circular Permutation
    \subsubsection*{Combination}
    Formulas of Combination, Combination of Identical Objects, Distrubution Based Questions, formation of Committee
    \subsubsection*{Probability}
    Classification of Events, Conditional Probability, Problems based on Coins, Dices and Cards, PnC based Probability questions
    \subsection*{Unit 6}
    \subsubsection*{Logical Reasoning 1}
    Visual Reasoning, Mirro/Water Image, Paper Cutting and Folding, Completion of Figure, Embedded Figure, Deviation of Figure
    \subsubsection*{Analytical Reasoning 1}
    Direction Sense: Questions Based on Determining Directions and Distance, Blood Relational: Questions Based on English Alphabet Problem, Question based on Word Problems
\end{multicols*}
\end{document}